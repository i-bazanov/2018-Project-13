\documentclass[12pt,twoside]{article}
\usepackage{jmlda}
\begin{document}
%\NOREVIEWERNOTES
\title
{Spherical CNN for QSAR prediction}
\author
{Dorokhin S.$^1$, Popova M.$^2$} % основной список авторов, выводимый в оглавление
\thanks
{Scientific advisor:  Strijov~V.\,V.
}
\organization
{$^1$MIPT, $^2$University of North Carolina}
\abstract
{
    \textbf{Abstract:}
    The task of predicting molecular properties e.g. biological activity or solubility based on the atomic structure is called QSAR (quantitative structure-activity relationship) prediction.
    It is a classical problem in drug design.
    Despite various algorythms (quantile regression, radial basis function neural networks) are an acceptable solution, there is still a need for more presice results. 
    A model originally developed for 3D shapes recognition was chosed and put under carefull examination in context of QSAR forecasting.
    That model is Spherical CNN, first suggested by Taco S. Cohen et. al., who managed to demonstrate that this NN performs well in several applications, including atomization energy prediction.
    The implemented model is compared with common CNN, RNN, graph CNN and Random Forest.    
\bigskip
\\
\textbf{Keywords}: \emph {QSAR prediction, Spherical CNN, drug design}
}
\maketitle
\end{document}
